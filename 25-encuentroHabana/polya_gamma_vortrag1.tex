
\documentclass{beamer}
\usepackage[utf8]{inputenc}
\usepackage[T1]{fontenc}
\usepackage{amsmath, amssymb, amsthm}
\usepackage{graphicx}
\usepackage{hyperref}

\title{Polya-Gamma Verteilung und Data Augmentation}
\subtitle{Anwendung in der Erdbeben-Vorhersage}
\author{Vortragender: [Dein Name]}
\date{\today}

\begin{document}

\begin{frame}
  \titlepage
\end{frame}

\begin{frame}{Inhalt}
  \tableofcontents
\end{frame}

\section{Einleitung}
\begin{frame}{Einleitung}
  \begin{itemize}
    \item Motivation: Warum Erdbeben-Vorhersage?
    \item Herausforderung: Komplexität und Unsicherheit in seismischen Daten
    \item Ziel: Einsatz der Polya-Gamma Verteilung zur Datenaugmentation
  \end{itemize}
\end{frame}

\section{Polya-Gamma Verteilung}
\begin{frame}{Was ist die Polya-Gamma Verteilung?}
  \begin{itemize}
    \item Einführung durch Polson, Scott und Windle (2013)
    \item Wird verwendet zur Vereinfachung von logistischen Modellen
    \item Eigenschaften:
    \begin{itemize}
      \item Symmetrische Verteilung
      \item Verwendbar in Gibbs-Sampling Algorithmen
    \end{itemize}
  \end{itemize}
\end{frame}

\begin{frame}{Mathematische Definition}
  Eine Zufallsvariable $X$ folgt einer Polya-Gamma Verteilung $PG(b, c)$, wenn ihre Dichte gegeben ist durch:
  \[ X \sim PG(b, c) \]
  \[ X = \frac{1}{2 \pi^2} \sum_{k=1}^{\infty} \frac{G_k}{(k-1/2)^2 + c^2/(4\pi^2)} \]
  wobei $G_k$ unabhängige Gamma-Verteilungen sind.
\end{frame}

\section{Data Augmentation mit Polya-Gamma}
\begin{frame}{Data Augmentation mit Polya-Gamma}
  \begin{itemize}
    \item Nützlich zur Vereinfachung von posterioren Verteilungen
    \item Ermöglicht effizienteres Gibbs-Sampling
    \item Anwendungsbeispiel: Binäre und zensierte Daten
  \end{itemize}
\end{frame}

\section{Anwendung in der Erdbeben-Vorhersage}
\begin{frame}{Erdbeben-Vorhersage mit Polya-Gamma}
  \begin{itemize}
    \item Seismische Daten: Hohe Unsicherheit und Nichtlinearität
    \item Modellierung der Auftretenswahrscheinlichkeit von Erdbeben
    \item Verwendung von logistischen Regressionsmodellen mit Polya-Gamma Data Augmentation
  \end{itemize}
\end{frame}

\begin{frame}{Vorteile und Herausforderungen}
  \begin{itemize}
    \item Vorteile:
    \begin{itemize}
      \item Stabilere Schätzung der Modellparameter
      \item Effizientere Sampling-Algorithmen
    \end{itemize}
    \item Herausforderungen:
    \begin{itemize}
      \item Hohe Rechenkosten bei großen Datensätzen
      \item Komplexität der Modellvalidierung
    \end{itemize}
  \end{itemize}
\end{frame}

\section{Fazit und Diskussion}
\begin{frame}{Fazit und Diskussion}
  \begin{itemize}
    \item Polya-Gamma Verteilung bietet elegante Lösung für logistische Modelle
    \item Vielversprechend in der Erdbeben-Vorhersage, aber herausfordernd
    \item Zukünftige Forschung:
    \begin{itemize}
      \item Optimierung der Rechenleistung
      \item Erweiterung auf komplexere seismische Modelle
    \end{itemize}
  \end{itemize}
\end{frame}

\begin{frame}{Fragen und Diskussion}
  \centering
  \Huge{Vielen Dank! \\ Fragen?}
\end{frame}

\end{document}
