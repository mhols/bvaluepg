
\documentclass{beamer}
\usepackage[utf8]{inputenc}
\usepackage[T1]{fontenc}
\usepackage{amsmath, amssymb, amsthm}
\usepackage{graphicx}
\usepackage{hyperref}

\title{Polya-Gamma Verteilung und Data Augmentation}
\subtitle{Anwendung in der Erdbeben-Vorhersage}
\author{Toni Luhdo}
\date{\today}

\begin{document}

\begin{frame}
  \titlepage
\end{frame}

\begin{frame}{Inhalt}
  \tableofcontents
\end{frame}

\section{Einleitung}
\begin{frame}{Einleitung}
  \begin{itemize}
    \item Motivation: Warum Erdbeben-Vorhersage?
    \item Herausforderung: Komplexität und Unsicherheit in seismischen Daten
    \item Ziel: Einsatz der Polya-Gamma Verteilung zur Datenaugmentation
  \end{itemize}
\end{frame}

\section{Polya-Gamma Verteilung}
\begin{frame}{Was ist die Polya-Gamma Verteilung?}
  \begin{itemize}
    \item Einführung durch Polson, Scott und Windle (2013)
    \item Wird verwendet zur Vereinfachung von logistischen Modellen
    \item Eigenschaften:
    \begin{itemize}
      \item Symmetrische Verteilung
      \item Verwendbar in Gibbs-Sampling Algorithmen
    \end{itemize}
  \end{itemize}
\end{frame}

\begin{frame}{Mathematische Definition}
  Eine Zufallsvariable $X$ folgt einer Polya-Gamma Verteilung $PG(b, c)$, wenn ihre Dichte gegeben ist durch:
  \[ X \sim PG(b, c) \]
  \[ X = \frac{1}{2 \pi^2} \sum_{k=1}^{\infty} \frac{G_k}{(k-1/2)^2 + c^2/(4\pi^2)} \]
  wobei $G_k$ unabhängige Gamma-Verteilungen sind.
\end{frame}

\begin{frame}{Herleitung der Polya-Gamma Verteilung}
  \begin{itemize}
    \item Ursprünglich entwickelt zur Behandlung von logistischen Likelihoods
    \item Verwendung der Identität:
    \[ \frac{\exp(\psi)}{1+\exp(\psi)} = \int_0^\infty \exp\left(-\omega \frac{\psi^2}{2}\right) p(\omega) d\omega \]
    \item Einführung der Polya-Gamma Verteilung zur Darstellung von $p(\omega)$
  \end{itemize}
\end{frame}

\begin{frame}{Mathematische Transformation}
  \begin{itemize}
    \item Transformation des logistischen Terms:
    \[ \frac{1}{(1+\exp(\psi))^b} = 2^{-b} \exp\left( -\frac{b \psi}{2} \right) \int_0^\infty \exp\left( -\omega \frac{\psi^2}{2} \right) p(\omega) d\omega \]
    \item Durch diese Transformation wird die Abhängigkeit von $\psi$ linearisiert
    \item Dadurch vereinfachte Berechnung im Rahmen von Gibbs-Sampling
  \end{itemize}
\end{frame}

\section{Data Augmentation mit Polya-Gamma}
\begin{frame}{Data Augmentation: Motivation und Idee}
  \begin{itemize}
    \item Ziel: Vereinfachung der posterioren Verteilungen
    \item Idee: Einführung latenter Variablen ($\omega$) zur linearen Darstellung
    \item Vorteile:
    \begin{itemize}
      \item Reduktion der Korrelation zwischen Parametern
      \item Beschleunigung der Konvergenz von Gibbs-Sampling
    \end{itemize}
  \end{itemize}
\end{frame}

\begin{frame}{Mathematische Herleitung der Data Augmentation}
  \begin{itemize}
    \item Logistische Likelihood:
    \[ p(y_i | \eta_i) = \frac{\exp(y_i \eta_i)}{1+\exp(\eta_i)} \]
    \item Einführung der latenten Variablen $\omega_i$:
    \[ p(y_i | \eta_i) = \int_0^\infty \exp\left(-\omega_i \frac{\eta_i^2}{2}\right) p(\omega_i) d\omega_i \]
    \item Dabei folgt $\omega_i \sim PG(1, \eta_i)$
  \end{itemize}
\end{frame}

\begin{frame}{Warum funktioniert Data Augmentation?}
  \begin{itemize}
    \item Polya-Gamma Variablen ermöglichen conjugate priors für logistische Modelle
    \item Führt zu bedingten Normalverteilungen für die Parameter
    \item Verbessert numerische Stabilität und Effizienz
  \end{itemize}
\end{frame}

\section{Anwendung in der Erdbeben-Vorhersage}
\begin{frame}{Erdbeben-Vorhersage mit Polya-Gamma}
  \begin{itemize}
    \item Seismische Daten: Hohe Unsicherheit und Nichtlinearität
    \item Modellierung der Auftretenswahrscheinlichkeit von Erdbeben
    \item Verwendung von logistischen Regressionsmodellen mit Polya-Gamma Data Augmentation
  \end{itemize}
\end{frame}

\begin{frame}{Vorteile und Herausforderungen}
  \begin{itemize}
    \item Vorteile:
    \begin{itemize}
      \item Stabilere Schätzung der Modellparameter
      \item Effizientere Sampling-Algorithmen
    \end{itemize}
    \item Herausforderungen:
    \begin{itemize}
      \item Hohe Rechenkosten bei großen Datensätzen
      \item Komplexität der Modellvalidierung
    \end{itemize}
  \end{itemize}
\end{frame}




\begin{frame}{...}
	...
\end{frame}

\section{Anwendung in der Erdbeben-Vorhersage}
\begin{frame}{Gutenberg-Richter-Gesetz}
	\begin{itemize}
		\item Beschreibt die Häufigkeit von Erdbeben in Abhängigkeit von der Magnitude $M$
		\item Mathematische Darstellung:
		\[ \log_{10} N(M) = a - bM \]
		\item $N(M)$: Anzahl der Erdbeben mit Magnitude $\geq M$
		\item Parameter:
		\begin{itemize}
			\item $a$: Maß für die seismische Aktivität einer Region
			\item $b$: Skalenfaktor, der die Häufigkeit großer vs. kleiner Beben beschreibt
		\end{itemize}
	\end{itemize}
\end{frame}

\begin{frame}{Bedeutung der Parameter $a$ und $b$}
	\begin{itemize}
		\item **Parameter $a$**:
		\begin{itemize}
			\item Bestimmt die absolute Anzahl der Erdbeben in einer Region.
			\item Höhere Werte von $a$ bedeuten mehr seismische Aktivität.
			\item Beeinflusst durch geologische Gegebenheiten.
		\end{itemize}
		\item **Parameter $b$**:
		\begin{itemize}
			\item Bestimmt das Verhältnis zwischen kleinen und großen Erdbeben.
			\item Typischerweise liegt $b$ zwischen 0.8 und 1.2.
			\item Niedrige Werte von $b$ deuten auf eine höhere Wahrscheinlichkeit großer Erdbeben hin.
		\end{itemize}
	\end{itemize}
\end{frame}

\begin{frame}{Visualisierung der Parameter $a$ und $b$}
	\centering
	\includegraphics[width=0.8\textwidth]{gutenberg_richter_plot.png} % Hier könnte eine Grafik eingefügt werden.
	\begin{itemize}
		\item Erhöhung von $a$: Mehr Erdbeben insgesamt
		\item Änderung von $b$: Steilere oder flachere Abnahme der Erdbebenhäufigkeit mit zunehmender Magnitude
	\end{itemize}
\end{frame}


\begin{frame}{Schätzung der Parameter mit Polya-Gamma}
	\begin{itemize}
		\item Modellierung der Erdbebenhäufigkeit als Poisson-Prozess:
		\[ N(M) \sim \text{Poisson}(\lambda(M)) \]
		\item Durch Log-Transformation entsteht ein logistisches Modell:
		\[ \log N(M) = a - bM + \epsilon \]
		\item $\epsilon$ ist ein Fehlerterm, der oft als Normalverteilung modelliert wird
		\item Die Polya-Gamma Verteilung kann zur Data Augmentation genutzt werden, um Gibbs-Sampling zu ermöglichen.
	\end{itemize}
\end{frame}

\begin{frame}{...}
	...
\end{frame}

\section{Fazit und Diskussion}
\begin{frame}{Fazit und Diskussion}
	\begin{itemize}
		\item Polya-Gamma Verteilung bietet elegante Lösung für logistische Modelle
		\item Vielversprechend in der Erdbeben-Vorhersage, aber herausfordernd
		\item Zukünftige Forschung:
		\begin{itemize}
			\item Optimierung der Rechenleistung
			\item Erweiterung auf komplexere seismische Modelle
		\end{itemize}
	\end{itemize}
\end{frame}


\begin{frame}{Definition der PG(b, c)-Verteilung}
	Die Pólya-Gamma Verteilung PG(b, c) ist eine exponentiell geneigte Version der PG(b,0)-Verteilung:
	
	
	$\omega \sim PG(b, c) \quad \text{wenn die Dichte gegeben ist durch}$
	
	
	$p(\omega | b, c) = \frac{\exp(-c^2 \omega / 2) p(\omega | b, 0)}{\mathbb{E}[\exp(-c^2 \omega / 2)]}$
	
	
	wobei  $p(\omega | b, 0)$ die Dichte der Standard-Pólya-Gamma-Verteilung  PG(b, 0)  ist.
\end{frame}




\begin{frame}{Fragen und Diskussion}
  \centering
  \Huge{Vielen Dank! \\ Fragen?}
\end{frame}

% Einführung
\begin{frame}{Einführung}
	\begin{itemize}
		\item Bayesianische Inferenz für logistische Modelle ist rechnerisch anspruchsvoll.
		\item Die Pólya-Gamma (PG) Verteilung ermöglicht eine effiziente Daten-Augmentierung.
		\item Ziel: Mathematische Grundlagen und Herleitungen für PG(b, c) verstehen.
	\end{itemize}
\end{frame}

% Definition der PG-Verteilung
\begin{frame}{Definition der Pólya-Gamma Verteilung}
	Eine Zufallsvariable \( \omega \) folgt einer \textbf{Pólya-Gamma Verteilung} \( \omega \sim PG(b, c) \), wenn sie die Dichte
	\begin{equation}
		p(\omega | b, c) = \frac{\exp(-c^2 \omega / 2) p(\omega | b, 0)}{\mathbb{E}[\exp(-c^2 \omega / 2)]}
	\end{equation}
	besitzt, wobei \( p(\omega | b, 0) \) die Standard-PG-Verteilung ist.
\end{frame}

% Laplace-Transformation
\begin{frame}{Laplace-Transformation der PG(b, c)-Verteilung}
	Die Laplace-Transformierte ist gegeben durch:
	\begin{equation}
		\mathbb{E} [e^{-t\omega}] = \frac{\cosh^{-b} (\sqrt{t}/2)}{\cosh^{-b} (c/2)}
	\end{equation}
	Dies zeigt, dass die PG-Verteilung als \textbf{unendliches Produkt von Gamma-Verteilungen} geschrieben werden kann.
\end{frame}

% Unendliche Faltung
\begin{frame}{Darstellung als unendliche Summe von Gammas}
	\begin{equation}
		\omega \sim \sum_{k=1}^{\infty} \frac{g_k}{(k - 1/2)^2 \pi^2 + c^2/4}, \quad g_k \sim \text{Gamma}(b, 1)
	\end{equation}
	Dies zeigt, dass die PG-Verteilung eine \textbf{unendliche Mischung von Gamma-Verteilungen} ist.
\end{frame}

% Erwartungswert und Varianz
\begin{frame}{Erwartungswert und Varianz}
	\begin{equation}
		\mathbb{E} [\omega] = \frac{b}{2c} \tanh(c/2)
	\end{equation}
	\begin{equation}
		\text{Var}(\omega) = \frac{b}{4c^3} (\sinh(c) - c)
	\end{equation}
\end{frame}

% Bedingte Verteilung für Gibbs-Sampling
\begin{frame}{Bedingte Verteilung für Gibbs-Sampling}
	\begin{equation}
		(\omega | \psi) \sim PG(b, \psi)
	\end{equation}
	Die Posterior-Verteilung der Koeffizienten ergibt sich als Normalverteilung:
	\begin{equation}
		(\beta | \omega, y) \sim N(\mu_\omega, \Sigma_\omega)
	\end{equation}
	mit:
	\begin{equation}
		\Sigma_\omega = (X^T \Omega X + B^{-1})^{-1}, \quad \mu_\omega = \Sigma_\omega (X^T \kappa + B^{-1} m)
	\end{equation}
\end{frame}

% Fazit
\begin{frame}{Fazit}
	\begin{itemize}
		\item Die Pólya-Gamma Verteilung ermöglicht effiziente Gibbs-Sampling-Algorithmen.
		\item Sie ist unter Addition geschlossen, was für hierarchische Modelle nützlich ist.
		\item Erwartungswerte und Varianz sind explizit berechenbar, was EM-Algorithmen erleichtert.
	\end{itemize}
\end{frame}

\end{document}
