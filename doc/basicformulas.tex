\documentclass{article}

\begin{document}

\section{The sigmoid function}

A convenient mapping from $R$ to $(0,1)$ is given by the sigmoid function
$$
\sigm(t) = \frac{e^t}{1+e^t} = \frac{1}{1+e^{-t}}
$$
We have
$$
\sigma(t) + \sigma(-t) = 1
$$
The derivative reads
$$
\sigma^\prime(t) = \sigma(t) \sigma(-t).
$$

\section{Model}

We consider a family of Poissonian observations like photon counts 
$$
n_i \sim \mbox{Poisson}_{\lambda_i},\quad i=1, \dots N
$$

The Poissonial frquencies $\lambda_i$ are parametrized by $f_i\in\mathbb{R}$
using the sigmoid function and an overall frequency $\lambda>0$
$$
\lambda_i = \lambda \sigma(f_i), \quad \sigma
$$
The Likelihood of the parameters $f_i$ given the observation of the $n_i$ then reads
$$
L(f_1, \dots, f_N | n_1,\dots, n_N) = C_{te} \prod_{i=1}^N \lambda_i \lambda_i^{n_i} e^{\lambda_i} 
$$

\end{document}
