\documentclass[11pt]{beamer}

% Theme / appearance
\usetheme{Madrid}
\usecolortheme{seahorse}
\setbeamertemplate{navigation symbols}{}
\setbeamertemplate{caption}[numbered]

% Encoding and language
\usepackage[utf8]{inputenc}
\usepackage[T1]{fontenc}
\usepackage[ngerman,english]{babel} % English main, German available

% Math and symbols
\usepackage{amsmath,amssymb,amsthm,mathtools}
\usepackage{bm}
\usepackage{siunitx}

% Graphics and tikz
\usepackage{graphicx}
\usepackage{tikz}
\usepackage{pgfplots}
\pgfplotsset{compat=1.18}

% Micro typography
\usepackage{microtype}

% Colors, links
\usepackage{xcolor}
% \usepackage[hidelinks]{hyperref}

% Useful shortcuts
\newcommand{\logit}{\operatorname{logit}}
\newcommand{\sigmoid}{\sigma}
\newcommand{\E}{\mathbb{E}}
\newcommand{\Var}{\operatorname{Var}}

% Title data — anpassen nach Bedarf
\title[Poisson \(\to\) Polya–Gamma]{From Poisson to Polya–Gamma in Spatial Models}
\author[T. Luhdo]{Toni Luhdo}
\institute[UP]{University of Potsdam / PG Project}
\date[Habana 2026]{Habana Encuentro, 2026}

% Optional: Outline before sections
\AtBeginSection[]{
  \begin{frame}{Outline}
    \tableofcontents[currentsection]
  \end{frame}
}

\begin{document}

\begin{frame}{TODOs}
  \begin{itemize}
    \item neue plots
    \item plots drehen
    \item andere Farbskala
    \item weniger sections
    \item references slide
    \item thx attention slide
    \item Figure_1 aufteilen
    % \item vielleicht ne Slide zu cholesky (wir berechnen nicht direkt die inverse)
  \end{itemize}
\end{frame}

\titlepage

\section{Motivation: Earthquake Intensity Field}

\begin{frame}{Earthquake Events as Point Data}
We observe earthquake epicenters as spatial point data.

\begin{center}
\includegraphics[width=0.8\textwidth]{figures/earthquakes_points.png}
\end{center}

Goal:
\begin{itemize}
\item Estimate a smooth spatial intensity field.
\item Identify high-risk regions.
\item Quantify uncertainty.
\end{itemize}
\end{frame}

\begin{frame}{From Points to Counts}
After gridding the region:

\[
n_i = \text{number of earthquakes in cell } i
\]

\begin{center}
\includegraphics[width=0.95\textwidth]{figures/Figure_1.png}
\end{center}

We now model:
\[
n_i \sim \text{Poisson}(\lambda_i)
\]
\end{frame}

\section{Poisson Model with Logistic Structure}

\begin{frame}{Basic Poisson Model}
\[
n_i \sim \text{Poisson}(\lambda_i)
\]

Instead of modeling $\lambda_i$ directly,
we introduce a latent field $f_i$:

\[
\lambda_i = \lambda \, \sigma(f_i),
\quad
\sigma(f) = \frac{1}{1+e^{-f}}
\]

\begin{itemize}
\item $f_i$ captures spatial structure
\item $\lambda$ is global intensity
\item $\sigma$ keeps rates bounded and stable
\end{itemize}
\end{frame}


\section{Likelihood Derivation}

\begin{frame}{Poisson Likelihood}
\[
p(n_i \mid f_i,\lambda)
=
\frac{(\lambda \sigma(f_i))^{n_i}}{n_i!}
\exp(-\lambda \sigma(f_i))
\]

Log-likelihood:
\[
\ell(f)
=
\sum_i
\left(
n_i \log \sigma(f_i)
-
\lambda \sigma(f_i)
\right)
\]

Problem:
\begin{itemize}
\item Non-quadratic in $f$
\item No conjugacy with Gaussian prior
\end{itemize}
\end{frame}


\section{Poisson Splitting Trick}

\begin{frame}{Introduce Latent Counts}
Define additional latent variables:

\[
k_i \sim \text{Poisson}(\lambda(1-\sigma(f_i)))
\]

Since

\[
\sigma(f) = \frac{e^f}{1+e^f}
\quad
1-\sigma(f)=\frac{1}{1+e^f}
\]

we obtain:

\[
p(n_i,k_i \mid f_i)
\propto
\frac{e^{n_i f_i}}{(1+e^{f_i})^{n_i+k_i}}
\]

Define:
\[
b_i = n_i + k_i
\]
\end{frame}


\section{Pólya--Gamma Identity}

\begin{frame}{Key Identity}
Polson–Scott–Windle identity:

\[
\frac{e^{\kappa f}}{(1+e^f)^b}
=
2^{-b}
\int_0^\infty
e^{-\frac{\omega f^2}{2}}
p(\omega \mid b,0)
\, d\omega
\]

with

\[
\kappa = n_i - \frac{b_i}{2}
\]

Conditional representation:

\[
\omega_i \mid f_i,b_i
\sim
\text{PG}(b_i,f_i)
\]
\end{frame}


\begin{frame}{Why This Is Powerful}
After introducing $\omega_i$:

\[
\log p(f_i \mid \omega_i)
=
-\frac12 \omega_i f_i^2
+
\kappa_i f_i
+
\text{const}
\]

This is quadratic in $f_i$.

\bigskip

Quadratic ⇒ Gaussian conditional posterior.
\end{frame}


\section{Gibbs Sampler}

\begin{frame}{Full Conditional Updates}

For each iteration:

\textbf{1) Latent counts}
\[
k_i \sim \text{Poisson}\left(\frac{\lambda}{1+e^{f_i}}\right)
\]

\textbf{2) Pólya–Gamma}
\[
\omega_i \sim \text{PG}(b_i,f_i)
\]

\textbf{3) Latent field}
\[
f \sim \mathcal{N}(m,Q^{-1})
\]

where

\[
Q = \Sigma^{-1} + \Omega
\]
\end{frame}

\begin{frame}{Why This Works So Well}
\begin{itemize}
\item No Metropolis tuning
\item Exact conditional updates
\item Fully Gaussian step for high-dimensional $f$
\item Works beautifully with sparse precision matrices
\end{itemize}
\end{frame}


\section{Application: Earthquake Data}

\begin{frame}{Observed Counts}
\includegraphics[width=0.99\textwidth]{figures/Figure_1.png}
\end{frame}

\begin{frame}{Estimated Rate Field}
\includegraphics[width=0.99\textwidth]{figures/estimated_rate_field.png}

Posterior mean:
\[
\hat{\lambda}_i = \lambda \, \sigma(f_i)
\]
\end{frame}


\begin{frame}{Takeaways}
\begin{itemize}
\item Poisson model with logistic link
\item Poisson splitting introduces logistic structure
\item Pólya–Gamma augmentation makes inference Gaussian
\item Efficient Gibbs sampling for spatial fields
\item Application: Earthquake intensity estimation
\end{itemize}
\end{frame}

\end{document}